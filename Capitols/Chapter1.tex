% !TEX root = ../main.tex

\chapter{Introducció, motivacions, propòsit i objectius del projecte}
\label{Chapter1}

\section{Introducció}
Aquest document presenta el desenvolupament tècnic de \textbf{NUMEN}, un sistema informàtic dissenyat per a l'automatització d'estudis numerològics complexos. El projecte aborda la transformació d'un procés manual, propens a errors i ineficiència, en una solució de programari robusta, escalable i precisa.

Addicionalment, el projecte transcendeix la simple eina de càlcul per oferir una estratègia de visibilitat externa mitjançant una plataforma pública. Aquesta vessant no només contextualitza els conceptes teòrics de la numerologia, sinó que està dissenyada per generar \textit{engagement}: presenta exemples reals d'estudis complets, ofereix una utilitat gratuïta (un "tastet") per al càlcul immediat del Camí de Vida —l'indicador base de la numerologia, que suposa una petita part del que és un estudi numerològic— i habilita un canal de comunicació directe amb el professional via WhatsApp, tancant així el cercle entre la curiositat de l'usuari i el servei expert.  A més, incorpora una gestió integral d'informes connectada a una base de dades que permet emmagatzemar, visualitzar i filtrar l'historial d'estudis, reduint el temps de procés d'hores a segons.

És important assenyalar que, tot i que l'origen d'aquest projecte té una forta component personal i subjectiva, s'ha pres la decisió estructural de separar aquestes motivacions de la memòria tècnica. Així doncs, el lector interessat en el rerefons més personal, la justificació filosòfica i l'origen de la idea pot consultar el \textbf{Pròleg} situat a l'inici d'aquest volum, on es troba la justificació més profunda, i l'Annex \ref{AppendixA} que conté més informació teòrica sobre la numerologia.  Aquesta distinció permet que els capítols següents es centrin exclusivament en l'enginyeria, l'anàlisi de requisits, el disseny de l'arquitectura i la implementació de la solució, mantenint el rigor i l'objectivitat propis d'un treball d'enginyeria informàtica.

Aquest treball pren com a referència fonamental l'obra \textit{La numerología a la luz del árbol de vida y las letras hebraicas} de Martine Coquatrix. No és una elecció trivial: aquesta metodologia integra la numerologia amb la càbala i el simbolisme de les lletres hebrees, oferint un sistema de càlcul i interpretació d'una complexitat algorítmica considerable, ideal per a ser modelat informàticament.

\section{Motivacions}
La motivació principal d'aquest projecte té una arrel profundament personal, però es justifica amb una problemàtica tècnica clàssica: la ineficiència dels processos manuals. La meva mare realitza estudis numerològics, i el que va néixer com una afició personal ha anat madurant fins a convertir-se en un servei que ofereix ocasionalment. Malgrat no ser la seva dedicació principal, cada encàrrec suposa un repte logístic, ja que la seva eina de treball ha estat, fins ara, el llapis, el paper i la calculadora.

La metodologia de Coquatrix és exhaustiva. La realització d'un sol estudi implica, de mitjana, una hora i mitja de dedicació exclusiva només per a la part matemàtica i de càlcul, sense comptar el temps posterior necessari per a la redacció d'informes personalitzats i la interpretació. Aquesta càrrega de treball manual feia coll d'ampolla, convertint el lliurament d'informes detallats en una tasca titànica i poc escalable, fins i tot per a un volum baix d'encàrrecs.

Com a enginyer, vaig identificar aquí una oportunitat clara. Es tractava d'un procés basat en regles matemàtiques definides, repetitiu i propens a l'error humà: el candidat perfecte per a l'automatització. Així, aquest PFG neix de la voluntat de posar l'enginyeria al servei d'una necessitat real, transformant un procés artesanal i tediós en una eina eficient, precisa i moderna.

\section{Propòsit}
El propòsit central del projecte és el desenvolupament de \textbf{NUMEN}, un assistent integral per a la realització d'estudis numerològics. El nom no és casual: en llatí, \textit{numen} fa referència a una "divinitat", "poder diví" o "esperit que presideix un lloc". Aquest terme captura l'essència del projecte, que busca erigir-se com un pont entre el món místic i intangible de la numerologia i el món lògic, estructurat i binari de la informàtica.

Més enllà de la simple automatització de càlculs, NUMEN pretén ser una reflexió sobre com la tecnologia pot interactuar amb disciplines humanístiques o esotèriques sense desvirtuar-les. El sistema no només agilitza processos, sinó que enriqueix l'experiència final mitjançant la visualització gràfica sobre la figura humana i l'ús d'Intel·ligència Artificial per assistir en la interpretació, creant una simbiosi entre la tradició ancestral i la tecnologia punta.

\section{Objectius}
Per tal de materialitzar aquest propòsit, s'han definit els següents objectius específics, que determinen l'abast funcional i tècnic del projecte:

\begin{textbox}{gray}{1}
\begin{itemize}
    \item \textbf{Documentació i implementació d'algoritmes:} Codificar els complexos algoritmes de càlcul numerològic seguint fidelment el mètode descrit per Martine Coquatrix, assegurant la precisió matemàtica.
    \item \textbf{Interfície d'usuari adaptada al model expert:} Dissenyar una interfície que no només sigui intuïtiva, sinó que s'adapti estrictament a la distribució visual definida per l'usuari expert. Això implica replicar digitalment la plantilla manual utilitzada tradicionalment, respectant al 100\% la disposició de les dades per mantenir la coherència amb el model mental de treball preexistent.
    \item \textbf{Visualització gràfica avançada:} Implementar un sistema per representar els resultats sobre una silueta humana, permetent una comprensió visual i immediata de les dades abstractes.
    \item \textbf{Disseny responsive i multiplataforma:} Garantir l'adaptabilitat de la interfície a qualsevol dispositiu (ordinador, tauleta o mòbil) mitjançant tècniques de disseny responsive. L'objectiu és assegurar que l'aplicació sigui accessible i plenament funcional des de qualsevol entorn, facilitant la consulta i creació d'estudis en mobilitat.
    \item \textbf{Integració d'Intel·ligència Artificial:} Incorporar serveis d'IA Generativa per assistir en la redacció i interpretació dels resultats, oferint un suport semàntic a les dades numèriques.
    \item \textbf{Gestió del coneixement i Human-in-the-Loop:} Implementar un sistema de persistència que permeti emmagatzemar, editar i refinar les respostes generades per la IA. Això inclou la captura de feedback dual: l'expert pot modificar el text directament (correcció tècnica), mentre que el client final pot valorar la interpretació a través d'un enllaç temporal segur (caducitat de 24h), creant així un dataset d'alta qualitat per al futur entrenament (Fine-Tuning) dels models.
    \item \textbf{Generació d'informes i consistència d'impressió:} Desenvolupar un motor de generació nativa de documents PDF basat en plantilles dinàmiques definides programàticament. Aquesta funcionalitat té la doble complexitat de garantir una sortida impresa vectorial d'alta fidelitat, independent de la resolució o mida de la pantalla del dispositiu, assegurant que el resultat final sigui professional i invariant.
    \item \textbf{Estratègia comercial i portal públic:} Desenvolupar un portal d'accés públic que serveixi d'aparador del servei. Aquest ha d'incloure recursos educatius sobre numerologia, una ``demo'' funcional simplificada (càlcul del Camí de Vida) i un canal de contacte directe (WhatsApp) per captar l'interès de nous usuaris sense requerir registre.
    \item \textbf{Seguretat i control d'accés:} Implementar un sistema robust d'autenticació (Login) que separi clarament l'entorn públic de l'àrea d'administració privada, garantint la confidencialitat de les dades sensibles dels estudis emmagatzemats. L'accés a les col·leccions privades de la base de dades es restringirà exclusivament als usuaris acreditats mitjançant l'aplicació estricta de regles de seguretat de Firestore (Security Rules). 
    \item \textbf{Validació amb casos reals:} Realitzar proves exhaustives utilitzant estudis reals previs per garantir que els resultats del programari coincideixen exactament amb els càlculs manuals experts.
\end{itemize}
\end{textbox}
