% !TEX root = ../main.tex

\chapter{Introducció, motivacions, propòsit i objectius del projecte}
\label{Chapter1}

\section{Introducció}
Numerologia. Una paraula que, d'entrada, pot semblar abstracta o fins i tot aliena en el context d'un Treball Final de Grau en Enginyeria Informàtica. Si jo fos el lector d'aquesta memòria, probablement em preguntaria: què hi fa una disciplina mil·lenària i esotèrica enmig de diagrames de classes i codi font? La resposta, potser, rau en la pròpia naturalesa dels finals i els principis.

Tancar aquesta etapa universitària amb la mateixa temàtica amb la qual vaig concloure el Batxillerat té un cert aire poètic, una mena de rima temporal que romantitza el pas d'estudiant a enginyer. Però més enllà del simbolisme, aquest projecte sorgeix d'una necessitat molt real: l'automatització de processos complexos en un entorn on la tecnologia encara no ha penetrat amb força.

Per situar-nos, cal definir la numerologia com una disciplina —sovint catalogada com a pseudociència— que estudia la relació mística entre els números, els éssers vius i les forces físiques o espirituals. Tot i no comptar amb l'aval del mètode científic tradicional, es fonamenta en la premissa que els números posseeixen propietats vibratòries capaces d'influir en la personalitat i el destí. Personalment, l'interès per aquest camp neix de l'observació empírica: malgrat la manca de base científica, els estudis numerològics sovint revelen patrons de conducta i trets de personalitat sorprenentment precisos, inclosos els propis. Aquesta "proximitat real percebuda" ha estat el catalitzador per voler aprofundir-hi des d'una òptica tècnica.

Aquest treball pren com a referència fonamental l'obra \textit{La numerología a la luz del árbol de vida y las letras hebraicas} de Martine Coquatrix. No és una elecció trivial: aquesta metodologia integra la numerologia amb la càbala i el simbolisme de les lletres hebrees, oferint un sistema de càlcul i interpretació d'una complexitat algorítmica considerable, ideal per a ser modelat informàticament.

\section{Motivacions}
La motivació principal d'aquest projecte té una arrel profundament personal, però es justifica amb una problemàtica tècnica clàssica: la ineficiència dels processos manuals. La meva mare realitza estudis numerològics. El que va néixer com una afició personal ha anat madurant fins a convertir-se en un servei que ofereix ocasionalment. Malgrat no ser la seva dedicació principal, cada encàrrec suposa un repte logístic, ja que la seva eina de treball ha estat, fins ara, el llapis, el paper i la calculadora.

La metodologia de Coquatrix és exhaustiva. La realització d'un sol estudi implica, de mitjana, una hora i mitja de dedicació exclusiva només per a la part matemàtica i de càlcul, sense comptar el temps posterior necessari per a la redacció d'informes personalitzats i la interpretació. Aquesta càrrega de treball manual feia coll d'ampolla, convertint el lliurament d'informes detallats en una tasca titànica i poc escalable, fins i tot per a un volum baix d'encàrrecs.

Com a enginyer, vaig identificar aquí una oportunitat clara. Es tractava d'un procés basat en regles matemàtiques definides, repetitiu i propens a l'error humà: el candidat perfecte per a l'automatització. Així, aquest TFG neix de la voluntat de posar l'enginyeria al servei d'una necessitat real, transformant un procés artesanal i tediós en una eina eficient, precisa i moderna.

\section{Propòsit}
El propòsit central del projecte és el desenvolupament de \textbf{NUMEN}, un assistent integral per a la realització d'estudis numerològics. El nom no és casual: en llatí, \textit{numen} fa referència a una "divinitat", "poder diví" o "esperit que presideix un lloc". Aquest terme captura l'essència del projecte, que busca erigir-se com un pont entre el món místic i intangible de la numerologia i el món lògic, estructurat i binari de la informàtica.

Més enllà de la simple automatització de càlculs, NUMEN pretén ser una reflexió sobre com la tecnologia pot interactuar amb disciplines humanístiques o esotèriques sense desvirtuar-les. El sistema no només agilitza processos, sinó que enriqueix l'experiència final mitjançant la visualització gràfica sobre la figura humana i l'ús d'Intel·ligència Artificial per assistir en la interpretació, creant una simbiosi entre la tradició ancestral i la tecnologia punta.

\section{Objectius}
Per tal de materialitzar aquest propòsit, s'han definit els següents objectius específics, que determinen l'abast funcional i tècnic del projecte:

\begin{textbox}{gray}{1}
\begin{itemize}
    \item \textbf{Documentació i implementació d'algoritmes:} Codificar els complexos algoritmes de càlcul numerològic seguint fidelment el mètode descrit per Martine Coquatrix, assegurant la precisió matemàtica.
    \item \textbf{Interfície d'usuari intuïtiva:} Dissenyar i desenvolupar una interfície que respecti el flux de treball dels professionals, fent que la tecnologia sigui transparent i facilitadora.
    \item \textbf{Visualització gràfica avançada:} Implementar un sistema per representar els resultats sobre una silueta humana, permetent una comprensió visual i immediata de les dades abstractes.
    \item \textbf{Integració d'Intel·ligència Artificial:} Incorporar serveis d'IA Generativa per assistir en la redacció i interpretació dels resultats, oferint un suport semàntic a les dades numèriques.
    \item \textbf{Gestió de dades i millora contínua:} Implementar una base de dades per emmagatzemar \textit{prompts} i respostes, establint les bases per a un futur entrenament i refinament dels models d'IA.
    \item \textbf{Generació d'informes:} Desenvolupar funcionalitats per a la creació i exportació automàtica d'informes complets, estètics i llestos per a ser lliurats al client.
    \item \textbf{Validació amb casos reals:} Realitzar proves exhaustives utilitzant estudis reals previs per garantir que els resultats del programari coincideixen exactament amb els càlculs manuals experts.
\end{itemize}
\end{textbox}
