% !TEX root = ../main.tex

\chapter{Marc de treball i conceptes previs}
\label{Chapter5}

\section{Entorn de Treball}

L'entorn de desenvolupament s'ha configurat per maximitzar la productivitat i la integritat del codi en un context de treball individual. Les eines principals utilitzades han estat:

\begin{itemize}
    \item \textbf{Sistema Operatiu:} Desenvolupament realitzat principalment sobre Windows.
    \item \textbf{IDE:} Visual Studio Code com a entorn de desenvolupament principal, equipat amb extensions per a Dart/Flutter i control de versions.
    \item \textbf{Control de Versions:} Ús de Git per a la gestió local del codi i GitHub com a repositori remot, seguint un flux de treball basat en \textit{main} i branques per funcionalitat.
    \item \textbf{Framework:} Flutter SDK (llenguatge Dart) per a la creació de l'aplicació multiplataforma.
    \item \textbf{Serveis al Núvol:} Gestió centralitzada mitjançant la consola de Firebase (Firestore, Hosting).
\end{itemize}

\section{Conceptes Previs: La Numerologia com a Domini}

Per comprendre la lògica de negoci implementada en \textbf{NUMEN}, és imprescindible definir els conceptes fonamentals de la numerologia segons el mètode de Martine Coquatrix. Des d'una perspectiva tècnica, podem considerar la numerologia com un sistema determinista de processament de dades, on les entrades (data de naixement i nom complet) es transformen mitjançant algoritmes específics en un conjunt de mètriques significatives.

A continuació es detallen els algoritmes i estructures de dades principals que conformen el nucli matemàtic de l'aplicació. Per a una descripció exhaustiva dels fonaments teòrics, la història de la numerologia i el significat detallat de cada número, consulteu l'\textbf{Annex \ref{AppendixA}}.

\subsection{Gematria i Conversió Alfanumèrica}
La base de tot càlcul numerològic textual és la conversió de lletres a valors numèrics. El sistema utilitzat assigna a cada lletra de l'alfabet llatí un valor de l'1 al 9, seguint un patró seqüencial.

La taula de conversió implementada és la següent:

\begin{table}[H]
    \centering
    \begin{tabular}{|c|c|c|c|c|c|c|c|c|}
        \hline
        \textbf{1} & \textbf{2} & \textbf{3} & \textbf{4} & \textbf{5} & \textbf{6} & \textbf{7} & \textbf{8} & \textbf{9} \\ \hline
        A & B & C & D & E & F & G & H & I \\ \hline
        J & K & L & M & N, Ñ & O & P & Q & R \\ \hline
        S & T & U & V & W & X & Y & Z &   \\ \hline
    \end{tabular}
    \caption{Taula de conversió Gematria (Sistema Pitagòric)}
    \label{tab:gematria}
\end{table}

És important destacar que caràcters com la `Ñ' s'assignen al valor 5 (igual que la `N'), i els accents o diacrítics s'ignoren (per exemple, `À' es tracta com `A').

\subsection{Reducció Teosòfica i Nombres Mestres}
La majoria de càlculs requereixen reduir qualsevol xifra superior a 9 a un sol dígit (de l'1 al 9). Aquest procés, conegut com a reducció teosòfica, consisteix a sumar els dígits del nombre de manera iterativa fins a obtenir-ne un d'elemental.

Matemàticament, per a un nombre $n$:
\[
R(n) = 
\begin{cases} 
n & \text{si } n \leq 9 \\
R(\sum_{d \in digits(n)} d) & \text{si } n > 9
\end{cases}
\]

\textbf{Excepció: Els Nombres Mestres}. Existeixen nombres especials que no s'han de reduir en determinats contextos: \textbf{11, 22 i 33}. Aquests es consideren ``Nombres Mestres'' i tenen una vibració superior. L'algoritme de reducció ha de contemplar aquesta excepció, verificant si el nombre resultant és mestre abans de continuar reduint.

\subsection{El Camí de Vida (Life Path)}
El Camí de Vida és una de les mètriques més rellevants, calculada a partir de la data de naixement. Representa la lliçó principal que l'individu ha vingut a aprendre.

L'algoritme de càlcul és el següent:
1. Es descompon la data en Dia, Mes i Any.
2. Es redueix cada component a un dígit (o nombre mestre).
3. Es sumen els tres components reduïts.
4. El resultat final es redueix novament si és necessari.

\subsection{La Inclusió (The Inclusion)}
La Inclusió és una eina complexa d'autoconeixement que analitza la distribució de les lletres del nom complet en nou àrees o ``cases''. Tècnicament, es tracta d'un histograma de freqüències on cada ``casa'' (de l'1 al 9) comptabilitza quantes vegades apareixen les lletres amb aquell valor numèric en el nom de la persona.

Aquest recompte genera el mapa d'\textbf{Habitants}:
\[ Habitants[i] = \text{count}(lletres \text{ amb valor } i) \quad \forall i \in [1,9] \]

A partir d'aquesta estructura base, es deriven altres vectors de dades mitjançant operacions matricials i lògiques:
\begin{itemize}
    \item \textbf{Matisos:} Relacionen els habitants entre si de manera recursiva.
    \item \textbf{Ponts:} Calculats com la diferència absoluta entre el valor de la casa i el seu habitant: $|Habitant[i] - (i+1)|$.
    \item \textbf{Evolució:} Indica cap a on tendeix l'energia de la casa.
    \item \textbf{Inconscient:} Revela patrons ocults basats en la posició de les lletres.
\end{itemize}

\subsection{Algorísmica del Nucli}
El cor de l'aplicació resideix en una sèrie d'algoritmes deterministes que transformen les dades d'entrada en mètriques significatives. A continuació, es detalla la lògica de negoci implementada per als càlculs més complexos.

\subsubsection{Càlculs Derivats del Nom}
A partir de la conversió gemàtrica, es calculen tres pilars fonamentals mitjançant la segregació de caràcters segons la seva tipologia fonètica:
\begin{itemize}
    \item \textbf{Ànima (Soul):} S'obté sumant exclusivament el valor de les \textbf{vocals} del nom complet. Representa la motivació interna.
    \item \textbf{Personalitat:} S'obté sumant exclusivament el valor de les \textbf{consonants}. Representa la projecció externa.
    \item \textbf{Expressió:} És la suma dels valors totals (Ànima + Personalitat) o de totes les lletres.
\end{itemize}

\subsubsection{Càlculs Derivats de la Data}
La data de naixement es processa per extreure'n cicles temporals, utilitzant operacions aritmètiques bàsiques però amb significats oposats:
\begin{itemize}
    \item \textbf{Desafiaments (Challenges):} Es calculen mitjançant la \textbf{resta} (valor absolut) entre els dígits reduïts del dia, mes i any (ex: $|Dia - Mes|$). Representen obstacles a superar.
    \item \textbf{Realitzacions (Pinacles):} Es calculen mitjançant la \textbf{suma} dels mateixos components. Representen els moments àlgids o metes.
\end{itemize}

\subsubsection{Algoritme del Número Llindar (NL)}
Un dels càlculs més complexos implementats és el del Número Llindar, el qual no segueix una fórmula aritmètica directa sinó una \textbf{lògica de prioritats condicional}. L'algoritme avalua l'histograma d'Habitants (Inclusió) en el següent ordre estricte:

\begin{enumerate}
    \item \textbf{Prioritat Kàrmica:} Si hi ha ``cases buides'' (habitants = 0), el NL és la suma d'aquestes cases buides.
    \item \textbf{Prioritat Dominant (Sobresurt):} Si un número apareix molt més que la resta (el seu valor és superior al segon màxim + 2), aquest esdevé el NL.
    \item \textbf{Prioritat per Repetició:} Si hi ha un empat en els valors màxims (més de 3 vegades repetit el màxim), es calcula una suma ponderada d'aquests.
    \item \textbf{Default (No Sobresurt):} Si no es compleix cap anterior, el NL és la suma de les dues cases amb major nombre d'habitants.
\end{enumerate}

\subsubsection{Càlcul d'Herències}
Les herències es calculen analitzant patrons específics en la distribució de la Inclusió, concretament observant la càrrega de les cases 1 (lideratge/pare) i 6 (família/responsabilitat), i com aquestes interactuen amb el total de lletres del nom.

\subsection{Algoritme de Visualització Dinàmica}
Una de les innovacions del sistema és la generació dinàmica de la figura representativa (l'home de Vitruvi) mitjançant la manipulació directa de vectors SVG.

\subsubsection{Lògica de Mapeig Vectorial}
La figura es descompon en regions (Cap, Braços, Cames, Tronc), i cadascuna està vinculada a coordenades numerològiques specífiques. L'algoritme recorre el mapa d'Habitants i la Data de Naixement i, per a cada coincidència, ``activa'' la regió corresponent del gràfic.

\subsubsection{Intensitat Cromàtica (Heatmap)}
No només es pinta la zona, sinó que es determina el color i la seva intensitat basant-se en la \textbf{freqüència d'aparició} del número:
\begin{itemize}
    \item \textbf{Escala de Vermells (Yang):} Utilitzada quan la coincidència prové directament dels dígits de la data de naixement (Dia, Mes, Any). A major repetició del dígit, més fosc i intens és el vermell (de \textit{LightSalmon} a \textit{DarkRed}).
    \item \textbf{Escala de Blaus (Yin):} Utilitzada quan la coincidència prové dels valors derivats de la Inclusió. Similarment, la intensitat del blau augmenta amb la freqüència (de \textit{PowderBlue} a \textit{Navy}).
\end{itemize}
Aquesta tècnica permet generar una representació visual única i immediata de l'energia del subjecte, on les zones fosques indiquen punts de màxima concentració energètica.

Aquests conceptes constitueixen la base teòrica sobre la qual s'ha construït el motor de càlcul de NUMEN.
