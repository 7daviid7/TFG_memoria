% !TEX root = ../main.tex

\chapter{Planificació}
\label{Chapter4}

\section{Estratègia de Planificació}
Per tal de gestionar la complexitat del projecte i assegurar el compliment dels objectius establerts, s'ha optat per dividir el desenvolupament en diferents Paquets de Treball (PT). Aquesta estratègia permet focalitzar els esforços en tasques concretes i mesurables, facilitant el seguiment del progrés i la detecció primerenca de possibles desviacions.

La metodologia de treball seguirà un enfocament iteratiu i incremental, especialment en les fases de desenvolupament i validació, permetent incorporar el feedback de manera contínua.

\section{Paquets de Treball}
D'acord amb la metodologia "Dos Temps" descrita anteriorment, els paquets de treball s'agrupen segons l'etapa de desenvolupament a la qual pertanyen.

\subsection{Etapa 1: Desenvolupament Artesanal (2024)}

\begin{table}[H]
    \centering
    \begin{tabular}{|p{0.25\textwidth}|p{0.70\textwidth}|}
        \hline
        \rowcolor{gray!20} \textbf{Nom del paquet} & \textbf{PT1: FONAMENTS I ALGORÍTMICA} \\ \hline
        \textbf{Descripció} & Estudi teòric i implementació manual del motor de càlcul (Etapa personal). \\ \hline
        \textbf{Tasques} & 
        \begin{itemize}[noitemsep,topsep=0pt,leftmargin=*]
            \item Estudi profund del llibre de Martine Coquatrix amb mentoratge expert.
            \item Disseny manual dels algoritmes de càlcul (sense IA).
            \item Implementació del motor matemàtic en Dart.
        \end{itemize} \\ \hline
        \textbf{Lliuraments} & Motor de càlcul funcional i validat. \\ \hline
         \textbf{Temps} & Agost 2024 - Octubre 2024 \\ \hline
    \end{tabular}
    \caption{PT1 - Etapa 1}
\end{table}

\begin{table}[H]
    \centering
    \begin{tabular}{|p{0.25\textwidth}|p{0.70\textwidth}|}
        \hline
        \rowcolor{gray!20} \textbf{Nom del paquet} & \textbf{PT2: PROTOTIPATGE FINAL} \\ \hline
        \textbf{Descripció} & Creació de la primera versió funcional de l'aplicació. \\ \hline
        \textbf{Tasques} & 
        \begin{itemize}[noitemsep,topsep=0pt,leftmargin=*]
            \item Desenvolupament de la interfície gràfica inicial.
            \item Sistema d'impressió basat en captures de pantalla.
            \item Validació de la funcionalitat bàsica.
        \end{itemize} \\ \hline
        \textbf{Lliuraments} & Aplicació "NUMEN v1" (ús local). \\ \hline
         \textbf{Temps} & Novembre 2024 - Desembre 2024 \\ \hline
    \end{tabular}
    \caption{PT2 - Etapa 1}
\end{table}

\subsection{Etapa 2: Professionalització (PFG 2025)}

\begin{table}[H]
    \centering
    \begin{tabular}{|p{0.25\textwidth}|p{0.70\textwidth}|}
        \hline
        \rowcolor{gray!20} \textbf{Nom del paquet} & \textbf{PT3: INFRAESTRUCTURA CLOUD} \\ \hline
        \textbf{Descripció} & Implementació de persistència i desplegament professional. \\ \hline
        \textbf{Tasques} & 
        \begin{itemize}[noitemsep,topsep=0pt,leftmargin=*]
            \item Configuració de Google Firebase Hosting.
            \item Implementació de la base de dades Firestore i \textbf{visualització de dades} (Historial).
            \item Definició de regles de seguretat inicials.
        \end{itemize} \\ \hline
        \textbf{Lliuraments} & Web App desplegada i BD operativa. \\ \hline
         \textbf{Temps} & Setembre 2025 (2 setmanes) \\ \hline
    \end{tabular}
    \caption{PT3 - Etapa 2}
\end{table}

\begin{table}[H]
    \centering
    \begin{tabular}{|p{0.25\textwidth}|p{0.70\textwidth}|}
        \hline
        \rowcolor{gray!20} \textbf{Nom del paquet} & \textbf{PT4: REFACTORITZACIÓ CORE} \\ \hline
        \textbf{Descripció} & Millora crítica de funcionalitats per a l'estàndard professional. \\ \hline
        \textbf{Tasques} & 
        \begin{itemize}[noitemsep,topsep=0pt,leftmargin=*]
            \item Migració a \textbf{Impressió PDF Nativa} (vectorial).
            \item Adaptació a Disseny Responsive (Mòbil).
            \item Implementació d'enllaços temporals (Sharing).
        \end{itemize} \\ \hline
        \textbf{Lliuraments} & Mòduls d'exportació i interfície millorats. \\ \hline
         \textbf{Temps} & Octubre 2025 (4 setmanes) \\ \hline
    \end{tabular}
    \caption{PT4 - Etapa 2}
\end{table}

\begin{table}[H]
    \centering
    \begin{tabular}{|p{0.25\textwidth}|p{0.70\textwidth}|}
        \hline
        \rowcolor{gray!20} \textbf{Nom del paquet} & \textbf{PT5: INTEL·LIGÈNCIA ARTIFICIAL} \\ \hline
        \textbf{Descripció} & Integració de la capa d'interpretació amb IA generativa. \\ \hline
        \textbf{Tasques} & 
        \begin{itemize}[noitemsep,topsep=0pt,leftmargin=*]
            \item Connexió amb API Google Gemini 2.5 Flash.
            \item Enginyeria de Prompts (protecció de PII).
            \item Integració de respostes a la interfície.
        \end{itemize} \\ \hline
        \textbf{Lliuraments} & Assistent d'interpretació operatiu. \\ \hline
         \textbf{Temps} & Novembre 2025 (3 setmanes) \\ \hline
    \end{tabular}
    \caption{PT5 - Etapa 2}
\end{table}

\begin{table}[H]
    \centering
    \begin{tabular}{|p{0.25\textwidth}|p{0.70\textwidth}|}
        \hline
        \rowcolor{gray!20} \textbf{Nom del paquet} & \textbf{PT6: PORTAL PÚBLIC I SEGURETAT} \\ \hline
        \textbf{Descripció} & Creació de l'entorn públic, estratègia de màrqueting i blindatge de seguretat. \\ \hline
        \textbf{Tasques} & 
        \begin{itemize}[noitemsep,topsep=0pt,leftmargin=*]
            \item Desenvolupament de la \textbf{Landing Page} educativa i mur de Testimonis.
            \item Implementació de la \textbf{Calculadora Pública} (Demo Camí de Vida).
            \item Sistema de \textbf{Login d'Administrador} i segmentació de rutes.
            \item Integració de contacte directe via \textbf{WhatsApp}.
            \item Refinament de \textbf{Firestore Security Rules} per protegir dades privades.
        \end{itemize} \\ \hline
        \textbf{Lliuraments} & Web pública operativa i sistema d'accés segur. \\ \hline
         \textbf{Temps} & Desembre 2025 (3 setmanes) \\ \hline
    \end{tabular}
    \caption{PT6 - Etapa 2}
\end{table}

\begin{table}[H]
    \centering
    \begin{tabular}{|p{0.25\textwidth}|p{0.70\textwidth}|}
        \hline
        \rowcolor{gray!20} \textbf{Nom del paquet} & \textbf{PT7: GESTIÓ I MEMÒRIA} \\ \hline
        \textbf{Descripció} & Documentació acadèmica i gestió del PFG. \\ \hline
        \textbf{Tasques} & 
        \begin{itemize}[noitemsep,topsep=0pt,leftmargin=*]
            \item Redacció de la memòria del projecte.
            \item Reunions de seguiment amb tutora.
            \item Preparació de la defensa.
        \end{itemize} \\ \hline
        \textbf{Lliuraments} & Memòria final del PFG. \\ \hline
        \textbf{Temps} & Transversal (Etapa 2) \\ \hline
    \end{tabular}
    \caption{PT7 - Etapa 2}
\end{table}

\section{Planificació Temporal}

\begin{figure}[H]
    \centering
    \caption{Fase 1: Desenvolupament Artesanal (2024)}
    \label{fig:gantt_phase1}
    \makebox[\textwidth][c]{
    \begin{ganttchart}[
        vgrid={*6{draw=none}, *1{dotted}},
        hgrid,
        x unit=0.075cm,
        y unit chart=0.9cm,
        time slot format=isodate,
        time slot unit=day,
        calendar week text={\currentweek},
        bar/.append style={fill=udgblue!50},
        bar height=0.6,
        link/.style={->, thick}
    ]{2024-08-01}{2024-12-31}
        \gantttitlecalendar{year, month=name} \\
        \ganttbar{PT1: Fonaments}{2024-08-01}{2024-10-31} \\
        \ganttbar{PT2: Prototipatge}{2024-11-01}{2024-12-31} \\
        \ganttlink{elem0}{elem1}
    \end{ganttchart}
    }
\end{figure}

\begin{figure}[H]
    \centering
    \caption{Fase 2: PFG i Professionalització (2025-2026)}
    \label{fig:gantt_phase2}
    \makebox[\textwidth][c]{
    \begin{ganttchart}[
        vgrid={*6{draw=none}, *1{dotted}},
        hgrid,
        x unit=0.075cm,
        y unit chart=0.9cm,
        time slot format=isodate,
        time slot unit=day,
        calendar week text={\currentweek},
        bar/.append style={fill=udgblue!50},
        bar height=0.6,
        link/.style={->, thick}
    ]{2025-09-01}{2026-01-31}
        \gantttitlecalendar{year, month=name} \\
        \ganttbar{PT3: Cloud Infra}{2025-09-01}{2025-09-15} \\
        \ganttbar{PT4: Refactor Core}{2025-09-15}{2025-10-31} \\
        \ganttbar{PT5: Int. Artificial}{2025-11-01}{2025-11-20} \\
        \ganttbar{PT6: Portal i Seguretat}{2025-12-01}{2025-12-21} \\
        \ganttbar{PT7: Memòria}{2025-09-01}{2026-01-20} \\
        \ganttlink{elem0}{elem1}
        \ganttlink{elem1}{elem2}
        \ganttlink{elem2}{elem3}
    \end{ganttchart}
    }
\end{figure}
