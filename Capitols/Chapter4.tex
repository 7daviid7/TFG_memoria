% !TEX root = ../main.tex

\chapter{Planificació}
\label{Chapter4}

\section{Estratègia de Planificació}
Per tal de gestionar la complexitat del projecte i assegurar el compliment dels objectius establerts, s'ha optat per dividir el desenvolupament en diferents Paquets de Treball (PT). Aquesta estratègia permet focalitzar els esforços en tasques concretes i mesurables, facilitant el seguiment del progrés i la detecció primerenca de possibles desviacions.

La metodologia de treball seguirà un enfocament iteratiu i incremental, especialment en les fases de desenvolupament i validació, permetent incorporar el feedback de manera contínua.

\section{Paquets de Treball}
% Translations for Gantt Chart
\languagealias{catalan}{Catalan}
\deftranslation[to=Catalan]{August}{Agost}
\deftranslation[to=Catalan]{September}{Setembre}
\deftranslation[to=Catalan]{October}{Octubre}
\deftranslation[to=Catalan]{November}{Novembre}
\deftranslation[to=Catalan]{December}{Desembre}
\deftranslation[to=Catalan]{January}{Gener}
\deftranslation[to=Catalan]{February}{Febrer}

A continuació es detallen els paquets de treball definits per al projecte \textbf{NUMEN}.

\begin{table}[H]
    \centering
    \begin{tabular}{|p{0.25\textwidth}|p{0.70\textwidth}|}
        \hline
        \rowcolor{gray!20} \textbf{Nom del paquet} & \textbf{PT1: GESTIÓ I DOCUMENTACIÓ} \\ \hline
        \textbf{Descripció} & Gestió integral del projecte, reunions de seguiment i redacció de la memòria. \\ \hline
        \textbf{Requisits/Tasques} & 
        \begin{itemize}[noitemsep,topsep=0pt,leftmargin=*]
            \item Planificació inicial i definició d'objectius.
            \item Reunions periòdiques amb la tutora.
            \item Redacció dels capítols de la memòria.
            \item Revisió i correcció de la documentació.
        \end{itemize} \\ \hline
        \textbf{Lliuraments} & Memòria del TFG, Actes de reunió. \\ \hline
        \textbf{Temps} & Transversal (Tot el projecte) \\ \hline
    \end{tabular}
    \caption{Paquet de treball: Gestió i Documentació}
    \label{tab:pt1}
\end{table}

\begin{table}[H]
    \centering
    \begin{tabular}{|p{0.25\textwidth}|p{0.70\textwidth}|}
        \hline
        \rowcolor{gray!20} \textbf{Nom del paquet} & \textbf{PT2: ANÀLISI I DISSENY} \\ \hline
        \textbf{Descripció} & Definició dels requisits del sistema i disseny de l'arquitectura i la interfície. \\ \hline
        \textbf{Requisits/Tasques} & 
        \begin{itemize}[noitemsep,topsep=0pt,leftmargin=*]
            \item Anàlisi de requisits funcionals i no funcionals.
            \item Estudi del mètode numerològic de Coquatrix.
            \item Disseny de la Base de Dades.
            \item Disseny de la Interfície d'Usuari (UI/UX).
        \end{itemize} \\ \hline
        \textbf{Lliuraments} & Diagrames UML, Mockups de la interfície, Esquema de la BD. \\ \hline
        \textbf{Temps} & 3 setmanes \\ \hline
    \end{tabular}
    \caption{Paquet de treball: Anàlisi i Disseny}
    \label{tab:pt2}
\end{table}

\begin{table}[H]
    \centering
    \begin{tabular}{|p{0.25\textwidth}|p{0.70\textwidth}|}
        \hline
        \rowcolor{gray!20} \textbf{Nom del paquet} & \textbf{PT3: NUCLI NUMEROLÒGIC} \\ \hline
        \textbf{Descripció} & Implementació dels algoritmes de càlcul segons el mètode de Coquatrix. \\ \hline
        \textbf{Requisits/Tasques} & 
        \begin{itemize}[noitemsep,topsep=0pt,leftmargin=*]
            \item Implementació de la conversió lletra-número (Gematria).
            \item Càlcul de l'Arbre de la Vida personalitzat.
            \item Implementació de càlculs de cicles i etapes de vida.
            \item Tests unitaris dels algoritmes matemàtics.
        \end{itemize} \\ \hline
        \textbf{Lliuraments} & Mòdul de càlcul numerològic funcional i testejat. \\ \hline
        \textbf{Temps} & 4 setmanes \\ \hline
    \end{tabular}
    \caption{Paquet de treball: Nucli Numerològic}
    \label{tab:pt3}
\end{table}

\begin{table}[H]
    \centering
    \begin{tabular}{|p{0.25\textwidth}|p{0.70\textwidth}|}
        \hline
        \rowcolor{gray!20} \textbf{Nom del paquet} & \textbf{PT4: DESENVOLUPAMENT DEL SISTEMA} \\ \hline
        \textbf{Descripció} & Construcció de l'aplicació completa, integrant Frontend, Backend i Base de Dades. \\ \hline
        \textbf{Requisits/Tasques} & 
        \begin{itemize}[noitemsep,topsep=0pt,leftmargin=*]
            \item Configuració de l'entorn i repositori.
            \item Desenvolupament de l'API Backend.
            \item Implementació del Frontend i visualització gràfica.
            \item Integració amb la Base de Dades.
        \end{itemize} \\ \hline
        \textbf{Lliuraments} & Codi font de l'aplicació, prototip funcional. \\ \hline
        \textbf{Temps} & 5 setmanes \\ \hline
    \end{tabular}
    \caption{Paquet de treball: Desenvolupament del Sistema}
    \label{tab:pt4}
\end{table}

\begin{table}[H]
    \centering
    \begin{tabular}{|p{0.25\textwidth}|p{0.70\textwidth}|}
        \hline
        \rowcolor{gray!20} \textbf{Nom del paquet} & \textbf{PT5: INTEL·LIGÈNCIA ARTIFICIAL} \\ \hline
        \textbf{Descripció} & Integració de serveis d'IA per a la interpretació de resultats. \\ \hline
        \textbf{Requisits/Tasques} & 
        \begin{itemize}[noitemsep,topsep=0pt,leftmargin=*]
            \item Disseny de \textit{prompts} per a la interpretació numerològica.
            \item Integració amb l'API d'OpenAI/Gemini.
            \item Emmagatzematge de respostes per a futur entrenament.
            \item Refinament de la qualitat de les respostes.
        \end{itemize} \\ \hline
        \textbf{Lliuraments} & Mòdul d'interpretació per IA integrat. \\ \hline
        \textbf{Temps} & 3 setmanes \\ \hline
    \end{tabular}
    \caption{Paquet de treball: Intel·ligència Artificial}
    \label{tab:pt5}
\end{table}

\begin{table}[H]
    \centering
    \begin{tabular}{|p{0.25\textwidth}|p{0.70\textwidth}|}
        \hline
        \rowcolor{gray!20} \textbf{Nom del paquet} & \textbf{PT6: VALIDACIÓ I PROVES} \\ \hline
        \textbf{Descripció} & Verificació del sistema amb casos reals i correcció d'errors. \\ \hline
        \textbf{Requisits/Tasques} & 
        \begin{itemize}[noitemsep,topsep=0pt,leftmargin=*]
            \item Proves d'integració i sistema.
            \item Validació manual amb la "client" experta (mare).
            \item Comparació de resultats automàtics vs manuals.
            \item Correcció de bugs i ajustos finals.
        \end{itemize} \\ \hline
        \textbf{Lliuraments} & Informe de proves, Versió final del programari. \\ \hline
        \textbf{Temps} & 2 setmanes \\ \hline
    \end{tabular}
    \caption{Paquet de treball: Validació i Proves}
    \label{tab:pt6}
\end{table}

\section{Planificació Temporal}
El desenvolupament del projecte es divideix en dues fases temporals diferenciades. A continuació es mostren els cronogrames corresponents a la fase inicial (2024) i la fase de finalització com a TFG (2025-2026).

\begin{figure}[H]
    \centering
    \caption{Fase 1. 
    Desenvolupament Inicial (2024)}
    \label{fig:gantt_phase1}
    \makebox[\textwidth][c]{
    \begin{ganttchart}[
        vgrid={*6{draw=none}, *1{dotted}},
        hgrid,
        x unit=0.075cm,
        y unit chart=0.9cm,
        time slot format=isodate,
        time slot unit=day,
        calendar week text={\currentweek},
        bar/.append style={fill=udgblue!50},
        bar height=0.6,
        group right shift=0,
        group top shift=0.7,
        group height=.3,
        group peaks width={0.2},
        link/.style={->, thick}
    ]{2024-08-01}{2024-12-31}
        \gantttitlecalendar{year, month=name} \\
        
        % Phase 1
        \ganttgroup[name=p1_total]{Fase 1: Des.}{2024-08-01}{2024-12-31} \\
        \ganttbar[name=p1_analisi]{Anàlisi i Requisits}{2024-08-01}{2024-08-31} \\
        \ganttbar[name=p1_disseny]{Disseny}{2024-09-01}{2024-09-15} \\
        \ganttbar[name=p1_nucli]{Nucli i Algoritmes}{2024-09-15}{2024-11-30} \\
        \ganttbar[name=p1_front]{Frontend Inicial}{2024-11-01}{2024-12-31} \\
        
        \ganttlink{p1_analisi}{p1_disseny}
        \ganttlink{p1_disseny}{p1_nucli}
        \ganttlink{p1_nucli}{p1_front}
    \end{ganttchart}
    }
\end{figure}

\begin{figure}[H]
    \centering
    \caption{Fase 2: Finalització TFG (2025-2026)}
    \label{fig:gantt_phase2}
    \makebox[\textwidth][c]{
    \begin{ganttchart}[
        vgrid={*6{draw=none}, *1{dotted}},
        hgrid,
        x unit=0.075cm,
        y unit chart=0.9cm,
        time slot format=isodate,
        time slot unit=day,
        calendar week text={\currentweek},
        bar/.append style={fill=udgblue!50},
        bar height=0.6,
        group right shift=0,
        group top shift=0.7,
        group height=.3,
        group peaks width={0.2},
        link/.style={->, thick}
    ]{2025-09-01}{2026-01-31}
        \gantttitlecalendar{year, month=name} \\
        
        % Phase 2
        \ganttgroup[name=p2_total]{Fase 2: TFG}{2025-09-01}{2026-01-31} \\
        \ganttbar[name=p2_gestio]{Gestió i Memòria}{2025-09-01}{2026-01-20} \\
        \ganttbar[name=p2_refactor]{Refactorització i BD}{2025-09-15}{2025-10-31} \\
        \ganttbar[name=p2_ia]{Integració IA}{2025-11-01}{2025-11-30} \\
        \ganttbar[name=p2_validacio]{Validació i Proves}{2025-12-01}{2025-12-20} \\
        
        \ganttlink{p2_refactor}{p2_ia}
        \ganttlink{p2_ia}{p2_validacio}
    \end{ganttchart}
    }
\end{figure}
