% !TEX root = ../main.tex

\chapter{Fonaments Teòrics de la Numerologia}
\label{AppendixA}

Aquest annex recull els fonaments teòrics que sustenten el sistema \textbf{NUMEN}, basats en la recerca realitzada per l'autor en el seu Treball de Recerca (TDR) i en l'obra de Martine Coquatrix.

\section{Què és la Numerologia?}

La numerologia és un conjunt de creences o tradicions que pretén establir una relació oculta entre els números, els éssers vius i les forces físiques o espirituals. També és una pràctica endevinatòria a través dels números \cite{wikipedia_numerologia}.

El seu estudi va ser popular entre els primers matemàtics, però no se la considera ja disciplina matemàtica. La comunitat científica fa temps que va relegar la numerologia a la categoria de pseudociència o superstició \cite{wikipedia_pseudociencia}, igual que l'astrologia pel que fa a l'astronomia, o l'alquímia, encara que aquesta última va tenir caràcter de protociència pel que fa a la química.

En numerologia, es diu que els números són un dels conceptes humans més perfectes i elevats. Segons els que la practiquen, la numerologia és la disciplina que pretén investigar la «vibració secreta» d'aquest codi i ensenyen a utilitzar els números en el seu benefici, per mitjà de l'estudi de la seva influència sobre persones, animals, coses i esdeveniments \cite{gonzalez_numerologia}.

L'any 530 a.C., Pitàgores, filòsof grec, va desenvolupar en forma metòdica una relació entre els planetes i la seva «vibració numèrica». Li va dir «la música de les esferes». Mitjançant el seu mètode de numerologia va afirmar que les paraules tenen un so que vibra en consonància amb la freqüència dels números com una faceta més de l'harmonia de l'univers i les lleis de la natura \cite{origen_numerologia}.

Com va dir Nikola Tesla: \textit{``Si vols conèixer els secrets de l'Univers, pensa en termes d'energia, freqüència i vibració''}. La física quàntica ha demostrat que som pura energia i informació vibrant a una determinada freqüència, i la numerologia estudia precisament la vibració dels números. Ens ensenya a sentir els números com una vibració.

\subsection{Origen de la Numerologia}
Encara que és difícil posicionar el seu origen amb precisió, avui podem identificar tres fonts o tres branques prominents:
\begin{itemize}
    \item \textbf{Numerologia Pitagòrica:} Pitàgores (500 aC) va ser qui va expandir aquest coneixement, fundant grups on s'estudiava el simbolisme dels números i la seva relació amb les lletres. Veia en pautes numèriques l'explicació dels fenòmens naturals i universals.
    \item \textbf{Simbolisme Cabalista:} Basat en la tradició mística jueva i l'Arbre de la Vida.
    \item \textbf{Simbolisme Cristià-Medieval:} Interpretació numèrica dels textos sagrats.
\end{itemize}

\section{L'Arbre de la Vida (Càbala)}

Per qualsevol qui defensi la numerologia, ``tot és número''. Els números representen l'armadura de l'univers, la seva defensa, el seu origen, la seva creació. L'Arbre de la Vida és un dels símbols cabalístics més importants del judaisme. Està compost per 10 esferes (sefirot) i 22 senders, cadascun dels quals representa un estat (sefira) que acosta a la comprensió de Déu i a la manera com ell va crear el món. La Càbala va desenvolupar aquest concepte com un model realista que representa un «mapa» de la Creació. Es considera la cosmologia de la Càbala \cite{wikipedia_arbol_vida}.

Alguns creuen que aquest «Arbre de la Vida» de la Càbala correspon a l'Arbre de la Vida esmentat a la Bíblia (Gènesi 2, 9) \cite{coquatrix_numerologia}.

\begin{figure}[H]
    \centering
    \includegraphics[width=0.4\textwidth]{Imatges/arbre_vida.png}
    \caption{Representació de l'Arbre de la Vida i els Sephiroth.}
    \label{fig:arbre_vida_tdr}
\end{figure}

Déu, en un estat de descans, decideix crear el món i projectar la seva força i la seva llum. Per fer-ho, ha d'imposar límits, passant de l'energia immensa de l'infinit a una energia definida. Així doncs, va formar aquest arbre de la vida compost per deu \textit{sephiroth} (recipients) que expressen deu essències o arquetips divins.

\subsection{Els Deu Sephiroth}
Cada sefira posseeix un nom que es pot comparar amb la seva ànima, i està associada a una jerarquia angèlica i a un planeta:

\begin{enumerate}
    \item \textbf{Kether (La Corona):} La voluntat primera. (Neptú / Serafins)
    \item \textbf{Hochmah (La Saviesa Revelada):} L'impuls dinàmic. (Urà / Querubins)
    \item \textbf{Binah (La Intel·ligència Concreta):} La forma que limita. (Saturn / Trons)
    \item \textbf{Hesed (La Misericòrdia i Abundància):} L'expansió. (Júpiter / Dominacions)
    \item \textbf{Geburah (El Rigor i la Força):} El límit i la disciplina. (Mart / Potències)
    \item \textbf{Tiferet (La Bellesa):} L'equilibri i l'harmonia. (Sol / Virtuts)
    \item \textbf{Netzah (La Victòria):} Les emocions i la natura. (Venus / Principats)
    \item \textbf{Hod (La Glòria):} L'intel·lecte i la comunicació. (Mercuri / Arcàngels)
    \item \textbf{Yesod (El Fonament):} L'inconscient i la memòria. (Lluna / Àngels)
    \item \textbf{Malkuth (El Regne):} El món físic i la realitat material. (Terra / Homes Perfectes)
\end{enumerate}

\textbf{Daat (El Coneixement):} És l'onzena sefira, misteriosa, la seu del coneixement. És el sol invisible, col·locat sobre el pilar central entre Kether i Tiferet. És el lloc de l'ànima.

En la part superior de l'Arbre de la Vida, per sobre de Kether, existeix un món infinit. Dintre d'aquest món, es troben diversos nivells d'existència negativa, que corresponen al 0; El Res i el Tot Possible:
\begin{itemize}
    \item \textbf{Ain Soph Aur:} Llum infinita.
    \item \textbf{Ain Soph:} Infinit, sense fi.
    \item \textbf{Ain:} Res, el punt més elevat.
\end{itemize}

Aquests tres nivells representen allò que transcendeix l'existència, la zona entre la Divinitat i la Creació. És l'absolut, el \textbf{NÚMERO DIVÍ PRIMORDIAL}, que ho conté tot i ho genera tot.

En aquesta existència negativa s'origina Kether, i a partir d'aquesta sefira, s'entra en la creació de l'univers. És com un flux d'energia divina que passa de sefira en sefira: de Kether passa per Hochmah, Binah, Hesed, Daat, Geburah, Tiferet, Netzah, Hod i finalment a Yesod, per acabar desembocant en Malkuth. En cada etapa, aquesta energia divina canvia de polaritat i d'aspecte. És com un riu que neix de Kether com una deu poderosa i cristal·lina que, mentre baixa per les diferents sefires, es va carregant de matèria i energia per acabar desembocant al mar, creant així EL MÓN \cite[p.~17]{coquatrix_numerologia}.


\subsection{Els Quatre Mons}
Segons la Càbala, l'Arbre es divideix en quatre mons:
\begin{itemize}
    \item \textbf{Azilout (Emanació):} Món dels arquetips (Kether). L'arrel dels mons.
    \item \textbf{Briah (Creació):} Món de l'esperit (Hochmah, Binah). Separació d'energies.
    \item \textbf{Yetzirah (Formació):} Món de la psique (Hesed a Yesod). Construcció i plasmació.
    \item \textbf{Assiah (Acció):} Món físic (Malkuth). On tenen forma els elements.
\end{itemize}

\subsection{El Cos Humà i l'Arbre}
El nostre cos està fet a imatge de l'Arbre de la Vida. De fet, portem a dintre de nosaltres l'esquema d'aquest arbre.
La columna vertebral correspon al pilar de l'equilibri, vinculant Kether amb Malkuth, la Corona amb el Regne, el cap amb els peus. Els dos costats del cos corresponen als dos pilars laterals, la Misericòrdia i el Rigor. En els tres pilars es recolzen els tres triangles ja esmentats. El triangle superior correspon al cap, la part del nostre cos oberta cap a l'infinit a través del xacra de la Corona. La Corona simbolitza la connexió de l'home amb el diví.
El cap té forma d'ou i representa allò que ha de néixer a la vida divina. En la iconografia hindú, el chakra Corona està representat per la flor de lotus de mil pètals que es desenvolupa al cim del crani. És per Kether per on el món diví baixa en l'home, el món de dalt, el «Mi», troba el món de baix, el «Ma» (Annick de Souzenelle). El segon triangle, format per Hochmah, Binah, Tiferet, està orientat cap avall i correspon a la regió cardíac-pulmonar. Aquí és on habita l'ésser espiritual, matriu de l'ésser diví, la nostra ànima. Entre el cap i el complex cardíac-pulmonar es troba el coll, comunicació entre l'ànima i l'obertura al diví. Quan Déu parla del seu «poble de la nuca rígida», li revela la ruptura de comunicació entre el cor i el cap, entre l'home i Déu. El tercer triangle invertit, que conté Netzah, Hod i Yesod, correspon al complex urogenital. Si es nodreix del món diví, l'home podrà obrir-se lentament a l'ésser espiritual que viu en ell. Si es nodreix només del món de baix i no es vincula amb l'ésser diví en ell mateix, coneixerà la mort: la matriu no haurà fructificat. La progressió en l'arbre es pot fer per diferents etapes. La primera etapa va de Malkuth fins al camí que uneix Hod a Netzah, és a dir, des dels peus fins als malucs i els ronyons. Correspon a les vèrtebres sacro-lumbars. En el desenvolupament de l'ésser correspon a la primera etapa, que és el naixement de l'ésser físic i l'etapa del «tenir». La segona etapa està continguda en el quadrilàter constituït per Hod, Netzah, Geburah i Hesed. Correspon al tronc, és a dir, a la part entre els malucs i les espatlles. És l'etapa de l'edat adulta. Correspon a les vèrtebres dorsals i al naixement de l'ésser espiritual. Quan l'home accedeix a aquesta etapa es troba en l'etapa de l'ésser. Es defineix aquesta etapa com el temps de pausa, de prova. Està formada pels quatre pilars, el 4 implica una noció d'estabilitat, de construcció sòlida, maduració i responsabilitat.
\section{Descripció i Simbolisme dels Números}

\begin{figure}[H]
    \centering
    \includegraphics[width=0.4\textwidth]{Imatges/numeros_tdr.png}
    \caption{Representació gràfica dels números.}
    \label{fig:numeros_tdr}
\end{figure}

\subsection{Yang i Yin}
\begin{itemize}
    \item \textbf{Yang (1, 3, 5, 7, 9):} Energia masculina, elements Foc i Aire. Són actius, analítics, racionals i competitius. Els agrada destacar i estar a primera fila.
    \item \textbf{Yin (2, 4, 6, 8):} Energia femenina, elements Aigua i Terra. Són intuïtius, flexibles, pacífics i col·laboradors. Privilegien la vida interior.
\end{itemize}

\subsection{Simbolisme Detallat}

\subsubsection{El Zero (0)}
El zero, és pot comparar amb un cercle que progressa fora del espai i el temps. És un vuit, que posseeix tot el potencial per ser omplert. Té un poder immens. Pot ser el tot i el no res. Com un ou, d'on neixen tots els altres números. Un número enigmàtic, considerat com a especial \cite{coquatrix_numerologia}.

En la reducció de la numerologia, és a dir, en la realització dels treballs numerològics, el zero no apareix, però potència el número resultant. PER EXEMPLE: 10 = 1 + 0 = 1. Però aquest 1, que prové del 10, és més potent.

\subsubsection{El Número 1}
És la representació de la unitat, l'origen del tot. La manifestació d'una força nova. Representa l'arquetip del \textbf{PARE}, referència psicològica molt important. És el pioner i el líder que obra els camins, que es llença amb decisió i autoritat als seus projectes. Injecta noves energies a les idees, i llença nous projectes intel·lectuals i mentals. Sempre vol ser el primer, i ser reconegut com el millor \cite{dela_numero_1}.

\begin{itemize}
    \item \textbf{COS HUMÀ:} Correspon a l'hemisferi cerebral esquerra, que determina el costat dret del nostre cos. És el centre de la voluntat, del potencial intel·lectual i abstracte, del pensament lògic, de l'acció.
    \item \textbf{TRAMPES DEL SEU EXCÉS:} Orgullós, dominador, egoista, ambiciós, impacient, irritable, tirànic, inflexible, intolerant, exigent, implacable, insensible.
\end{itemize}

\subsubsection{El Número 2}
De la unitat de l'1, passem a la dualitat del 2. Representa el principi femení. El dos, neix de l'1, i forma amb ell un sistema binari. D'aquesta manera, tenim dos pols: El Yang i el Yin \cite{calcuworld_numero_2}.

La grafia d'aquest número, està composta per corbes i línies rectes, per la qual cosa, indica la seva adaptabilitat i flexibilitat. És caracteritza per tenir una actitud de disponibilitat, sempre atent per escoltar i ser flexible. El número 2, vol ser i sentir, lentament, descobrint la seva afectivitat i sentiments interiors profunds. Té com a objectiu, donar i rebre.

És el número de la feminitat i maternitat. Correspon a l'arquetip de mare. Vol estimar i ser estimat, envoltar-se d'amor, viure al nivell dels seus sentiments, de la seva tendresa. Té por a la soledat, i sempre està buscant la fusió amb algú altre. A més, sempre compte amb la capacitat d'adaptar-se i d'actuar amb molta intuïció i diplomàcia. A través de la seva delicadesa, de la seva virtut d'escoltar i de poder acollir, pot arribar a canviar el curs de les coses i les opinions dels altres. ÉS EL NÚMERO DE LA COL·LABORACIÓ.

Posseeix un gran potencial artístic gràcies a la seva sensibilitat i apertura al món de la imaginació i els somnis. No té confiança en si mateix, i necessita sempre la aprovació d'algú altre per poder creure amb ell mateix. Sol viure amb complexa d'inferioritat, a no ser que sigui reconegut per algú altre. És incapaç de suportar els conflictes, i per això prefereix callar i no expressar el que sent, per temor a ser rebutjat pels altres. S'empassa les seves emocions, fins al punt d'asfixiar-se i destruir-se.

\begin{itemize}
    \item \textbf{COS HUMÀ:} Correspon a l'hemisferi cerebral dret, el món de la imaginació, la sensibilitat i la intuïció. Aquest hemisferi, determina la part esquerra del nostre cos, la part femenina i receptiva. Està relacionada també amb els òrgans dobles del cos, és a dir, els pulmons, els ronyons...
    \item \textbf{TRAMPES DEL SEU EXCÉS:} Dubte, dualitat, complexa d'inferioritat, falta d'autoestima i autodestrucció, vulnerabilitat, descontrol de les emocions, dependència, rol de víctima i inseguretat.
\end{itemize}

\subsubsection{El Número 3}
El 3, és la unió de l'1 i el 2. És un número perfecte, perquè conté la força de l'1 (energia masculina), i la sensibilitat del 2 (energia femenina). Representa l'arquetip del nen, fruit de la unió del pare i la mare \cite{euroresidentes_numero_3}.

La seva grafia està composta per corbes sensuals, amb tres puntes obertes que reflecteixen l'intercanvi amb els altres:
\begin{itemize}
    \item En el cap: la comunicació verbal.
    \item En el cor: la comunicació sentimental i emocional.
    \item En el ventre: La comunicació corporal i el plaer.
\end{itemize}

\begin{figure}[H]
    \centering
    \includegraphics[width=0.4\textwidth]{Imatges/tres.png}
    \caption{Representació del número 3.}
    \label{fig:numero_3}
\end{figure}



Té l'art de comunicar-se i expressar-se a través de la seva veu, la seva paraula. Posseeix un do per ensenyar i transmetre el seus coneixements. En el pla artístic, pot emprendre qualsevol projecte, degut al seu talent, originalitat i el seu desenvolupat sentit estètic. Necessita estimar i ser estimat per TOTS. Li agrada complaure i destacar. Ser reconegut pel seu públic. La admiració dels altres és la seva força, però també pot convertir-se en la seva debilitat.

Quan no se sent estimat i acceptat pels altres, pot tancar-se en si mateix, sofrir molt i tornar-se amargat i frustrat. Necessita estar envoltat pels seus amics, dels seus germans, de la seva família. Del seu grup de comunicació per compartir i gaudir de la vida en companyia. Està molt pendent de la mirada dels altres, i la necessita per tenir confiança en si mateix. LA SEVA IMATGE SOCIAL ÉS PRIMORDIAL, i quan aquest aspecte és torna obsessiu, pot arribar a convertir-se en una persona superficial.

Té l'art de viure bé el present. L'AQUÍ I L'ARA. És molt generós, li agrada compartir, és molt alegre, optimista i espontani. Adora les festes i les reunions, i posseeix un gran sentit de l'humor. El seu públic li resulta tant indispensable, que pot passar la seva vida actuant en rols determinats, sense viure la seva pròpia identitat. Té dificultats per viure en soledat, ja que tem enfrontar-se a la seva pròpia veritat.

\begin{itemize}
    \item \textbf{TRAMPES DEL SEU EXCÉS:} Frivolitat, superficialitat, agitació, dispersió, falta d'identitat, irresponsabilitat, vanitat, inestabilitat.
\end{itemize}

\subsubsection{El Número 4}
El seu element és la terra, que la transforma a través del seu treball. És el número de la encarnació. La seva grafia és un conjunt de rectes, amb angles ben definits, que inspiren fermesa i seguretat. Una imatge que recorda al tro d'un rei \cite{celeste_numero_4}.

El seu símbol, és el quadrat o el cub. Ens inspira estabilitat, equilibri i força. L'arquetip del 4 són les arrels. Representa la base sobre la qual construïm la nostra vida. La seva força radica en les normes establertes. Abans de donar un pas, necessita saber cap a on va dirigit. Per aquesta raó, a vegades avança molt lentament i li falta flexibilitat. És molt fidel, i sempre necessita aferrar-se a les tradicions de la seva terra natal i de la seva família. Com a conseqüència, se sent responsable del patrimoni material familiar. És el número del treball i la organització. Li agrada el treball ben fet, i dedicarà tots els seus esforços per complir els seus deures. Ha de tenir en compte, però, a no omplir-se de càrregues extres.

\begin{itemize}
    \item \textbf{COS HUMÀ:} Representa la part dura del cos, és a dir, l'esquelet i les dents.
    \item \textbf{TRAMPES DEL SEU EXCÉS:} Bloqueig mental i emocional, melancolia, rigidesa, obsessió per l'ordre, tosquedat, materialisme, obsessió per el treball, avarícia, inflexibilitat, dogmatisme i poca flexibilitat mental.
\end{itemize}

\subsubsection{El Número 5}
S'ubica a mig camí entre l'1 i el 9. Té l'energia exterior de l'1, obrint camins i lluitant en busca de nous reptes, i l'energia interior del 9, per la seva necessitat de reflexió. Està associat a Mart, el Déu de la guerra. La seva imatge és la d'un rei energètic que va a la batalla a fi de conquistar altres regnes: VIURE INSEGURETATS I CÓRRER RISCOS \cite{celeste_numero_5}.

El seu símbol és el pentagrama, i la seva grafia consisteix en una barra horitzontal superior que representa el seu gran desenvolupament mental, la seva curiositat sense límits i la seva gran capacitat d'anàlisi. Aquesta barra, recolzada sobre una corba, representa l'energia vital i sexual. Per un número 5, el vital és sentir-se lliure de llaços innecessaris. Necessita galopar lliurement sense sentir-se empresonat sota convencionalismes i regles. És rebel amb el seu mode de pensar i viure. És conscient del poder de la seva energia, i ha d'aprendre a dominar-la i controlar-l. Així mateix, ha de mantenir fermes les rendes del seu cavall, amb molta disciplina i rigor, pel contrari, acabaria desbocat.

Sempre busca experiències noves pel pur plaer d'experimentar, però, encara que a vegades cremi les seves ales per volar massa a prop del sol, aquest sempre està llest per cobrar el preu dels seus errors. És molt honest, i sempre juga net. El número 5, ha de saber, que també té dificultats, doncs mai acaba els seus projectes personals, degut a que sempre s'impacienta i comença a interessar-se per altres projectes. Córrer el risc de convertir-se en un ``picaflor'', que va de flor en flor en busca del no res. Per aquesta raó, és essencial que s'escolti a si mateix i mediti les seves decisions. És una aventurer per naturalesa, i sempre està disposat a viure tot tipus d'aventures i reptes físics/intel·lectuals. Se sent permanentment atret pels viatges i els descobriments d'altres cultures, però sempre necessita tornar als seus inicis. Degut a això, té el do dels idiomes.

Aquesta força vital, el pot portar a excessos: hiperactivitat, consumisme cultural... pot caure al descontrol, abusar de l'alcohol i el tabac, les drogues, realitzar experiències perilloses... el seu temperament explosiu el porta a situacions de risc. Per això, és important que el número cinc realitzi esport, per calmar aquesta hiperactivitat i equilibrar la seva energia.

\begin{itemize}
    \item \textbf{COS HUMÀ:} Representa el mode de viure la nostre part masculina Yang i la nostre energia vital, canalitzada a través de l'energia sexual o creativa. Representa el KI, el nostre centre energètic.
    \item \textbf{TRAMPES DEL SEU EXCÉS:} Irritabilitat, violència, agressivitat, inconstància, llibertinatge, addiccions (sexe, alcohol, drogues...)
\end{itemize}

\subsubsection{El Número 6}
És el centre de l'amor. El 6, es representa amb dos triangles encastats que formen el segell de Salomó, l'estrella de sis puntes \cite{celeste_numero_6}.

La grafia del sis, és una única corba que s'embolcalla sobre si mateixa. La seva base rodona, representa un ventre, simbolisme de la fertilitat i de l'amor. Viu segons la seva sensibilitat i les seves emocions. És el número de la feminitat que aporta pau i harmonia. Ens convida al descans, a la flexibilitat i la apertura del cor. La seva prioritat i desig és donar, protegir i cuidar als altes. Per contra, li costa molt donar i rebre. Sovint arribar al sacrifici, oblidant-se de si mateix. Per evitar aquesta situació, el número sis ha d'amar-se i cuidar-se en primer lloc, per poder entregar un amor equilibrat i sa. Inclús, pot arribar a ser possessiu i manipular afectivament als altres, a fi d'aconseguir ser amat i amar. Pot arribar a tenir pànic ha ser abandonat pel seus éssers estimats.

Se sentirà satisfet en professons que estan relacionades amb el cos i la salut, com per exemple medicina, estètica, massatgista... A més, posseeix un gran poder creatiu, i és capaç d'apreciar la bellesa visible i invisible dels sers.

L'arquetip del sis, representa la forma de viure la nostre part femenina Yin. Degut a la seva gran sensibilitat és bastant vulnerable, i sol viure bastantes crisis d'identitat i desequilibris emocionals forts. A més, també pot caure en l'apatia i la comoditat, conformant-se i sen mandrós.

\begin{itemize}
    \item \textbf{COS HUMÀ:} Representa el plexe cardíac, el cor, el lloc de la nostre ànima. El centre de les emocions.
    \item \textbf{TRAMPES DEL SEU EXCÉS:} Vulnerabilitat, fragilitat, possessió, manipulació, gelós, sacrifici, frustració, culpabilitat, obsessió, depressió, apatia i mandra.
\end{itemize}

\subsubsection{El Número 7}
És el número de la espiritualitat. És l'enllaç entre l'humà i el diví. El número sagrat per excel·lència. Sol associar-se a la bellesa i a la perfecció, sovint també, a les victòries. El 7, ens introdueix en la bellesa espiritual, simbolitzada en el Sabbat (sèptim dia), quan Déu descansa per admirar la perfecció de la creació \cite{celeste_numero_7}.

El 7 ens convida a trobar la unitat interior després d'un llarg aprenentatge personal que requereix autocontrol, exigència i una profunda concentració.

La seva grafia s'assembla al número 1, però amb un cap més desenvolupat, més sobrecarregat que sembla inclinar-se degut al seu pes de saber i coneixement. La seva base es fràgil i poc estable.

Té un sentit de la bellesa i de l'estètica que l'impulsa a buscar permanentment l'elegància. És molt sensible a la naturalesa i necessita estar en contacte amb ella. Necessita saber i aprendre, recolzant-se en llibres, escrits o documents. Tota la seva vida està buscant coneixements. Degut a això, és molt savi. Avança amb constància per aconseguir donar respostes a les grans interrogacions i enigmes existencials, descobrint les grans veritats. És el número dels grans místics, teòlegs i filòsofs que ens permeten accedir a una realitat superior. L'excés del treball mental, però, pot ocasionar-li dificultats per viure les contingències materials i expressar els seus sentiments. És difícil per aquesta persona, connectar-se amb el seu cos i el seu cor.

Necessita la soledat i el silenci a l'hora de meditar, contemplar i reflexionar. No obstant, això pot desencadenar a tancar-se en la seva ``torre de marfil'', i aïllar-se dels altres. Pot arribar a viure amb massa serietat i austeritat. Aquest excés de saber, pot provocar en ell actituds d'orgull, superioritat, menyspreu i sarcasme.

Si no aconsegueix els seus ideals, pot caure en una depressió o procés d'autodestrucció aïllant-se per complet del món.

\begin{itemize}
    \item \textbf{COS HUMÀ:} El número 7, està relacionat amb l'hemisferi cerebral esquerra, el lloc de la voluntat, la lògica i l'anàlisi. Correspon al chakra del tercer ull, que ens permet l'apertura espiritual.
    \item \textbf{TRAMPES DEL SEU EXCÉS:} Intransigència, intolerància, sarcasme, supèrbia, soledat, aïllament, pessimisme, auto-càstig, depressió i desequilibri mental.
\end{itemize}

\subsubsection{El Número 8}
És la manifestació de la intel·ligència perfecta i la capacitat de transformar la matèria. Aquest caduceu, està compost per dos parts simètriques que ens ajuden a comprendre millor el simbolisme del número 8 \cite{euroresidentes_numero_8}:
\begin{itemize}
    \item La serpentina de l'esquerra, representa el que rebem de l'univers: talents i dons.
    \item La serpentina de la dreta correspon a la manera en que retornem a l'univers els fruits de la Terra transformada.
    \item La vara del centre simbolitza la mediació entre Déu i l'Home.
\end{itemize}

El 8 és el símbol del Kundalini, de l'energia sexual, de la creativitat que ens ajuda a entrar en altres nivells de consciència. Com es pot veure en la seva grafia, les energies circulen des de dalt fins a baix, com si fos un rellotge de sorra. Representa el símbol de l'infinit en posició vertical.

El 8, representa el nostre poder, el nostre talent. A més, el 8, per ser dos vegades 4, és sòlid, responsable i té un gran sentit del concret. Té una gran personalitat d'organització i realització. És un número que comparteix aspectes Yin i Yang. El seu aspecte Yin, l'ajuda a prendre's el seu temps, preparar estratègies i madurar els seus projectes abans de llançar-se a l'acció, mentre que el seu aspecte Yang, l'ajuda amb una gran capacitat d'organització, rendiment i producció.

Per ser reconegut pels altres, necessita afirmar el seu poder material i econòmic. No pot viure sense treballar. Li agraden els reptes, i els afronta amb coratge i audàcia. Es capaç de realitzar qualsevol projecte. Darrere aquesta aparença exigent i autoritària, és generós i bondadós quan se sent amat pels altres. A més, posseeix un gran sentit de la justícia i l'honestedat.

Aquesta aparença exterior de poder i seguretat, pot amagar una gran fragilitat interior. És un número tant fort i amb tanta energia, que fàcilment pot cometre excessos de poder, orgull i dominació.

\begin{itemize}
    \item \textbf{COS HUMÀ:} Representa el plexe solar, i tots els seus aspectes de potència i vulnerabilitat.
    \item \textbf{TRAMPES DEL SEU EXCÉS:} Orgull, materialisme, cobdícia, supèrbia, arrogància, tirania, despotisme, autoritarisme, agressivitat, violència, crueltat i impaciència.
\end{itemize}

\subsubsection{El Número 9}
El número 9 és l'apertura de la ment i l'esperit. És l'últim número d'una sola xifra, que ens proposa anar més enllà dels nostres límits, i obrir-nos camins cap a horitzons més amplis. Al tarot, és el símbol de l'ermità, el que viu i manté la seva llum interior. El número Maestre interior, que necessita la seva llibertat física i espiritual per transmetre la saviesa al món \cite{celeste_numero_9}.

La seva grafia és un 6 invertit, i tanmateix, com el 6, està format únicament per corbes que evoquen les relacions amb aspectes sensibles i receptius. La diferència entre el 6 i el 9, doncs, mentre el 6 viu els seus sentiments i les seves emocions, el 9 exposa totes les seves energies al seu cap. La seva debilitat, és tanmateix la seva virtut. És com un globus que està molt carregat d'idees, somnis... però que té una connexió fràgil i poc estable amb la terra.

Per ser 3 vegades 3, viu en el seu món imaginari, per la qual cosa, també posseeix un gran creativitat i sensibilitat artística molt aguda.

El número del coneixement i misticisme per la seva connexió directe entre el seu cor i l'energia còsmica. El número del visionari. Compte amb una compassió que li permet captar els missatges de patiment que emeten els altres, i sentir la necessitat d'ajudar-los. Un número altruista, amb una vocació humanitària; viu amb l'ideal d'ajudar als més dèbils. El número de la compassió i la tolerància.

És tant idealista, compassiu i generós, que pot arribar a l'abnegació, oblidant-se de si mateix, posicionant als altres com la seva única prioritat. A més, si no aconsegueix el seu objectiu d'ajudar als altres, pot arribar a caure en l'autodestrucció i depressió. El 9, té problemes per acceptar les regles i exigències de la vida quotidiana, perquè prefereix refugiar-se al seu món imaginari.

\begin{itemize}
    \item \textbf{COS HUMÀ:} Representa el chakra corona, i la connexió amb els mons superiors, degut a la seva descendència en Yesod.
    \item \textbf{TRAMPES DEL SEU EXCÉS:} Utopia, falta de connexió amb la realitat, inconsistència, aïllament, mandra, marginació, depressió, comportaments agressius, orgull, dominació i autodestrucció.
\end{itemize}

\subsection{Els Nombres Mestres (11, 22, 33)}
Són números compostos per deu xifres idèntiques. Tenen una energia especial que requereix grans esforços i disponibilitats per part de la persona que el viu. Aquests números mestres, acompanyes a l'evolució de la humanitat, i actualment poden viure l'11, el 22 i el 33 \cite{adivinario_numeros_maestros}.

Són regales o carismes que ens ofereix l'univers per construir-nos, créixer i posar-nos al servei de la comunitat.

Aquests números mestres, són la simplificació dels números de l'1 al 9. És a dir, són números amb una potència molt més elevada. Per exemple:
\begin{itemize}
    \item 11 = 1 + 1 = 2 $\rightarrow$ l'onze, reduït, és un 2, i és manifesta com a tal, quan aquest aconsegueix viure en harmonia els aspectes del 2. Quan aquest, aconsegueix llimar les dureses de les trampes del seu excés, quan aconsegueix viure en pau i felicitat amb si mateix.
    \item 22 = 2 + 2 = 4
    \item 33 = 3 + 3 = 6
\end{itemize}

Aquests números mestres, que representen vibracions molt elevades, és viuen en certs moments de la vida, en certs cicles o anys, propers a l'evolució del ser. Però, aquests necessiten estar sostinguts per números energèticament forts, com l'1, el 5 o el 8.

No són números fàcils de viure, per les seves potents vibracions, la persona els ha d'acceptar, i ser conscient de la seva energia, no caure en la superioritat. Són números que ens indiquen que tenim un potencial més elevat, per col·laborar amb l'evolució de la terra i ajudar als altres.

\subsubsection{EL NÚMERO 11: ``Missatger Diví''}
Està compost per dos vegades 1, per la qual cosa, necessita ser reconegut i afirmar-se amb autoritat. A la vegada, representa el dos (1+1=2), la saviesa de l'arbre de la vida. Per aquest motiu, se l'anomena el missatger diví.

S'assembla al número 9, però amb una missió més elevada. Apareix en busca d'una veritat profunda. Està dotat de força moral fora del comú, així com una intel·ligència subtil i un potent magnetisme intel·lectual.

Però aquest número pot tenir també els seus riscos i trampes, degut a la seva barreja de dos vegades 1, i del 2:
\begin{itemize}
    \item Vulnerabilitat emocional.
    \item Nervis i impaciència.
    \item Un camí de vida que ha de recórrer amb saviesa i evitant els excessos.
    \item Complexa de superioritat, excés d'autoritat o manipulació.
\end{itemize}

\subsubsection{EL NÚMERO 22: ``Constructor de Futurs''}
És presentat com un inventor apassionat, un visionari amb una gran sensibilitat, un organitzador genial. El seu objectiu és realitzar grans projectes, amb idees noves i futuristes. És el número de la superació per aconseguir propòsits més elevats. Però el 22 pot tenir també els seus límits:
\begin{itemize}
    \item La pèrdua de l'equilibri psicològic quan és deixa portar per corrents contradictòries.
    \item La mala gestió de les emocions. És el punt dèbil del 22. Pot viure tensions afectives i conflictes racionals.
    \item L'excés de tensions que prové d'una energia física mal distribuïda o mal utilitzada.
\end{itemize}

\subsubsection{EL NÚMERO 33: ``L'amor incondicional''}
Quan ens trobem amb un 33, és necessari veure si la persona és capaç de viure l'aspecte del 6 amb harmonia i equilibri. El 33 és el número del sacrifici, de l'amor perfecte, de la compassió i la comprensió.

El número dels grans mesters. Poques persones son capaces de viure'l amb equilibri perquè suposa un gran discerniment per evitar caure en l'excés de l'abnegació i sacrifici.

\section{La Inclusió}

La inclusió és un camí d'autoconeixement a través de nou cases, o nou aspectes de la vida. Cada casa estarà ocupada per un ``habitant'', que estarà en evolució constant. Aquests ``habitants'' transmeten els missatges qualitatius i ens indiquen de quina forma viurem aquests aspectes de la vida. Representen les eines de base que tenim al néixer.

\subsection{Com construir la Inclusió}

Hem d'assignar a cada lletra el seu valor, d'acord amb el quadre següent:

\begin{table}[H]
    \centering
    \begin{tabular}{|c|c|c|c|c|c|c|c|c|}
        \hline
        \textbf{1} & \textbf{2} & \textbf{3} & \textbf{4} & \textbf{5} & \textbf{6} & \textbf{7} & \textbf{8} & \textbf{9} \\
        \hline
        A & B & C & D & E & F & G & H & I \\
        J & K & L & M & N, Ñ & O & P & Q & R \\
        S & T & U & V & W & X & Y & Z & \\
        \hline
    \end{tabular}
    \caption{Taula de conversió lletra-número per a la Inclusió.}
    \label{tab:inclusio_lletres}
\end{table}

Per construir la inclusió, es compta quantes lletres hi ha de cada valor numèric en el nom complet (noms i cognoms).

EXEMPLE: 
\begin{itemize}
    \item \textbf{Nom:} DAVID
    \item \textbf{Cognom pare 1:} QUINTANILLA
    \item \textbf{Cognom mare 1:} JIMENEZ 
    \item \textbf{Cognom pare 2:} GARCIA
    \item \textbf{Cognom mare 2:} MOLINA
\end{itemize}

\begin{figure}[H]
    \centering
    \includegraphics[width=0.8\textwidth]{Imatges/exemple.png}
    \caption{Exemple de la inclusió amb el nom DAVID QUINTANILLA}
    \label{fig:numero_exemple}
\end{figure}

\subsection{Les 9 Cases o les Nou Àrees de la Vida}

\begin{itemize}
    \item \textbf{CASA 1:} L'estructura de l'ego, identitat en el món, dinàmica d'afirmació, relació amb el pare.
    \item \textbf{CASA 2:} La forma de viure les emocions, capacitat d'escoltar i acollir, relació amb la mare, feminitat.
    \item \textbf{CASA 3:} Relació amb els altres (germans, amics), creativitat, expressió personal, el nen interior.
    \item \textbf{CASA 4:} Món material, treball, cos físic, bases personals (història familiar), compromís.
    \item \textbf{CASA 5:} Adaptació al canvi, capacitat d'anàlisi, llibertat, energia vital i sexualitat.
    \item \textbf{CASA 6:} Relacions afectives, responsabilitat familiar, amor a un mateix, benestar, harmonia.
    \item \textbf{CASA 7:} Espiritualitat, reflexió, coneixement, esquemes mentals, herències culturals.
    \item \textbf{CASA 8:} Talents, estratègia, realització, poder material, estatus social.
    \item \textbf{CASA 9:} Consciència universal, saviesa interior, servei humanitari, inconscient.
\end{itemize}

\subsection{Els Habitants: Formes de matisar les Cases}

Els habitants ens permetran entendre com o de quina manera vivim cada aspecte del nostre ser, a través de les nou cases.

\begin{itemize}
    \item \textbf{HABITANT 1:} Autònoma, amb autoritat, innovadora. Risc: Impaciència, dominació, orgull.
    \item \textbf{HABITANT 2:} Sensible, tendre, diplomàtica. Risc: Dubte, inseguretat, pors afectives.
    \item \textbf{HABITANT 3:} Creativa, comunicativa, alegre. Risc: Superficialitat, dispersió, necessitat de reconeixement.
    \item \textbf{HABITANT 4:} Estructurada, responsable, pacient. Risc: Rigidesa, pors materials, excés de responsabilitat.
    \item \textbf{HABITANT 5:} Dinàmica, lliure, aventurera. Risc: Inestabilitat, agressivitat, dispersió.
    \item \textbf{HABITANT 6:} Harmoniosa, servicial, responsable. Risc: Possessivitat, sacrifici, culpabilitat.
    \item \textbf{HABITANT 7:} Reflexiva, profunda, perfeccionista. Risc: Intolerància, aïllament, supèrbia.
    \item \textbf{HABITANT 8:} Eficient, constructiva, estratègica. Risc: Dominació, materialisme, agressivitat.
    \item \textbf{HABITANT 9:} Intuïtiva, idealista, compassiva. Risc: Desconnexió de la realitat, frustració, utopia.
\end{itemize}
