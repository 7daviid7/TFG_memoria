\chapter*{Pròleg}
\addcontentsline{toc}{chapter}{Pròleg}
\label{Prologue}

Numerologia. Una paraula que, d'entrada, pot semblar abstracta o fins i tot aliena en el context d'un Treball Final de Grau en Enginyeria Informàtica. Si jo fos el lector d'aquesta memòria, probablement em preguntaria: què hi fa una disciplina mil·lenària i esotèrica enmig de diagrames de classes i codi font? La resposta, potser, rau en la pròpia naturalesa dels finals i els principis.

Tancar aquesta etapa universitària amb la mateixa temàtica amb la qual vaig concloure el Batxillerat té un cert aire poètic, una mena de rima temporal que romantitza el pas d'estudiant a enginyer. Si bé la motivació darrere d'aquesta elecció és sincera i personal, el projecte transcendeix la simple motivació personal. Sorgeix d'una necessitat molt real: l'automatització de processos complexos en un entorn on la tecnologia encara no ha penetrat amb la força necessària.

Entenc que l'elecció d'aquesta temàtica, de caràcter pseudocientífic, pugui generar cert escepticisme en un entorn acadèmic rigorós. Per això, vull explicitar que el focus d'aquest treball no és validar la numerologia ni redescobrir-la des d'una vessant filosòfica —qüestió que, de fet, ja vaig intentar abordar en el treball de recercar (TDR) com a estudiant de Batxillerat— sinó donar resposta a una necessitat real d'un usuari, que en aquest cas és la meva mare.  La meva tasca com a enginyer és abstreure'm del debat sobre la matèria i aplicar les eines informàtiques necessàries per automatitzar un procés manual, garantint-ne la precisió i l'eficiència tècnica.


Dit això, què és exactament la numerologia? És la disciplina que estableix una relació mística entre els nombres, els éssers vius i les forces espirituals. Postula que els nombres tenen propietats vibratòries capaces d'incidir en la realitat i, mitjançant analogies, ens permeten comprendre millor el funcionament de l'univers. En essència, és l'art d'interpretar símbols vibracionals. (l'Annex \ref{AppendixA} conté més informació).

\begin{reflectionbox}
Com bé deia Nikola Tesla: ``si vols conèixer els secrets de l'Univers, pensa en termes d'energia, freqüència i vibració''. En l'actualitat, la física moderna ens recorda que som energia i informació vibrant. La numerologia, en el seu nucli, fa justament això: estudiar la vibració intrínseca dels nombres. Ens convida a sentir-los no com a xifres fredes, sinó com a freqüències. Per això, en l'intent d'aprofundir en la nostra cultura emocional, sovint trobem en els nombres una eina sorprenentment precisa.
\end{reflectionbox}

Tot i no comptar amb l'aval del mètode científic clàssic, la disciplina es fonamenta en la premissa que els nombres posseeixen qualitats vibratòries que influeixen en la personalitat i el destí. En el meu cas, l'interès neix d'una observació empírica, gairebé íntima: malgrat l'absència de validació científica, els estudis numerològics sovint destapen patrons de conducta i trets de personalitat d'una precisió desconcertant. Aquesta ``proximitat real percebuda'' ha estat el motor que m'ha impulsat a revisitar la matèria, primer des d'una òptica filosòfica i, ara, des d'una perspectiva purament tècnica.
